\documentclass[12pt,a4paper]{article}
% \usepackage{fullpage}



\usepackage{graphicx}
\usepackage{lipsum}

% Maths formatting
\usepackage{amsmath,amsthm,amsfonts,amssymb,amscd}
\usepackage{mathtools}
\usepackage{mathrsfs}
\usepackage{bm}

% Table packages
\usepackage{booktabs}
\usepackage{multirow}
\usepackage{tabularx}
\usepackage{longtable}
\usepackage{makecell}

% Figure captioning
\usepackage{caption}
\usepackage{subcaption}




\begin{document}
	

\title{Suggested Structure}

\maketitle

\section{Intro}
[Let's fill this in last.]

\subsection{Related Literature}
[It wouldn't hurt to do a bit of a literature review. For instance, 1-2 pages on (i) ins and outs of unemployment, (ii) displacement costs (including methodologies) and (iii) labor market policies.]

\section{The Ins and Outs of (Voluntary) Unemployment}
[Here, I think, we could put an extended version of the micro note.]

\subsection{Data and Methodology}
[We need to explain where the data is from, but especially what the definitions are -- what exactly is voluntary unemployment and how it is measured. Then, we can put in a short description of what we'll do, which I think can boil down to some simple in-out analysis ala Shimer.]

\subsection{(Voluntary) Unemployment in the Data}

\paragraph{Basic patterns.} [Bit of an extended version of the micro-note. Perhaps starting with unemployment overall showing that (I assume) voluntary unemployment doesn't matter that much. But then zooming in on youth unemployment which changes the picture.]

\paragraph{Ins and outs.} [A Shimer/Fujita/Ramey type of exercise to understand how much voluntary separations matter. Ideally also the variance decompositions over the business cycle to get some headline number: e.g. voluntary separations account for XYZ percent of youth unemployment. But also, we should zoom in on the job finding rates of voluntarily separated people - are they very different?]

\section{Drivers of Voluntary Unemployment}
[Here we could move on to trying to understand what voluntary separations really are.]

\subsection{Individual Characteristics}
[Document individual characteristics of (voluntary) separators. This is much of the work you've done so far in terms of comparing samples. But I'd much less focus on the nitty gritty and instead try to paint broad strokes in terms of education/tenure/income (distribution) etc.]

[Probably would want to do this for the sample as a whole and zooming in on the young (acknowledging the difficulties of doing so)]

\subsection{Life-Cycle profiles and Separation Penalties}
[Having documented that (voluntary) separators are quite similar, we move on to what happens to them after they separate. Here we bring forward the idea of comparing volutnary to involuntary separators and to J2J movers.]

\paragraph{Life-Cycle profiles.} [Zooming in on the young, document life-cycle profiles of (voluntary) separators -- wages, employment histories, separation and job finding rates etc. Are they more like involuntary ones or J2J movers?]

\paragraph{Separation penalties.} [This would be the wage penalties (but potentially also other penalties -- cummulated employment times/tenure/future separations?). I'd view this as ``just'' another way to describe the data, rather than making a case for exogenous variation and causal identification. All the discussion about samples, different methodologies would -- I would suggest -- go into the appendix as robustness.]

\section{Structural Framework}
[Acknowledging the difficulties of identification, we'd build a simple ``toy'' model. Perhaps partial equilibrium, but ideally with some features that we feel may be important -- e.g. one option is to include match specific and worker specific productivity. While the former would be up to search, the latter accumulates over time. Voluntarily separating is then a trade-off between forgoing human capital accumulation in return for the option value of a better match. But I guess we can come up with other/additional trade-offs. The model could then at least tell us some parameter ranges which would make the predictions consistent with the data and hopefully these are informative. It'd be even better if, using the model, we figure out that we should look into some particular patterns in the data which we haven't yet tried. ]



	

\end{document}
