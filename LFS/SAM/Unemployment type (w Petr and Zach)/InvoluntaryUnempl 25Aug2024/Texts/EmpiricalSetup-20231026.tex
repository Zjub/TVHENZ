
\documentclass[12pt,a4paper]{article}
%%%%%%%%%%%%%%%%%%%%%%%%%%%%%%%%%%%%%%%%%%%%%%%%%%%%%%%%%%%%%%%%%%%%%%%%%%%%%%%%%%%%%%%%%%%%%%%%%%%%%%%%%%%%%%%%%%%%%%%%%%%%%%%%%%%%%%%%%%%%%%%%%%%%%%%%%%%%%%%%%%%%%%%%%%%%%%%%%%%%%%%%%%%%%%%%%%%%%%%%%%%%%%%%%%%%%%%%%%%%%%%%%%%%%%%%%%%%%%%%%%%%%%%%%%%%
\usepackage{amsfonts, amsmath, amsthm,mathtools}
\usepackage{graphicx}
\usepackage{caption}
% \usepackage{undertilde}
\usepackage{bbm}
\usepackage{rotating}
\usepackage{multirow}
\usepackage{xcolor}
\usepackage{amsmath}
\usepackage{booktabs}
\usepackage{natbib}
\usepackage{appendix}
\usepackage{setspace}
\usepackage{soul}
\usepackage[utf8]{inputenc}
\usepackage{marginnote}
\usepackage[margin=1in]{geometry}
%\usepackage[top=1.5cm, bottom=1.5cm, outer=2cm, inner=2cm, heightrounded, marginparwidth=4.0cm, marginparsep=.5cm]{geometry}
\usepackage[normalem]{ulem}
\usepackage{titletoc}
\usepackage{enumerate}
\usepackage{comment}
\usepackage{xr}
\usepackage{colortbl}
%\usepackage{ntheorem}
\usepackage[unicode=true]{hyperref}
\onehalfspacing
%\setstretch{1.2}

\begin{document}
	
What about estimating the following?
\begin{equation}
y_{i,t} = \alpha + \sum_g D^g \left( \sum_{a} \delta_a^g D^a\right) + \Lambda X_{i,t} + \epsilon_{i,t},
\end{equation} 	

\noindent where $y_{i,t}$ is some outcome variable (e.g. hours, job satisfaction tenure etc.), $g$ indicate groups of individuals (voluntary separators, involuntary separators, J2J and remainers as you've defined them), $D^g$ are the associated group dummies, $D^a$ are age dummies and $X_{i,t}$ are control variables (probably most importantly education, occupation, gender and anything else really that the data has and could help explain the LHS variable). 

Then, the coefficients of interest are the estimated life-cycle profiles $\delta_a^g$'s for different groups. I think that without controlling for anything (i.e. no $X_{i,t}$), you should simply recover what you have. In some sense, we could use this to tease out where some of the effects are coming from. For instance, thinking about wages - ``unconditionally'' there seems to be this penalty of separating. However, perhaps individuals don't differ in regards to their wages once we take into account tenure (and that even voluntary separators ``restart'' their tenure). 



\end{document} 